\section{Introduction}

Fast Fourier Transform (\fft) algorithms are among the most important of the XXth
Century. Applications range widely from engineering to pure science.
For example: image and sound signal analysis and lossy compression.
jpg and mp3 are file formats that encode the signal using its Fourier modes.
In astrophysics \fft\ is used to solve the differential field
equations of Newtonian Gravity and General Relativity as well, in huge N-body
simulations of the cosmological evolution of the universe. While the Fourier
modes of those fields, usually the matter density, encode statistical estimators
that link theoretic models and observation. \cite{adamek_2016,springel_2020} are
examples of some of the most recent such simulations.

There are many different \fft\ algorithms as well as many implementations of
those. The most notable of such implementations is the \fftw\ \cite{FFTW05}
free-software library for C language, which boasts itself of being the fastest
implementation at least for CPUs---note there exists 
implementations for GPUs such as
\cufft\footnote{\url{docs.nvidia.com/cuda/pdf/CUFFT_Library.pdf}}.
Despite \fft\ being well known and widely used, there is no C++ library 
counterpart of \fftw,
and to our knowledge there is not a single \fft\ implementation templated on the
type.

This \gsoc\ project aimed to implement a new \fft\ library
within \boostmath\footnote{\url{www.boost.org/libs/math}}---a large mathematical
library that
offers high-performance functions of pure and applied mathematics
but no \fft\ support before this date. 
