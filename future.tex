\section{Summary of features and future outlook}
\boostfft\ is still in phase of development. This year's \gsoc\ has helped to
kickstart with the project and the authors had made huge steps in order to
present the library as it is today, such that basic functionalities like one
dimensional \dft s are possible with theoretically optimal complexity. 
Let us briefly list the features of the \boostfft\ in its current state:
\begin{itemize}
    \item the library brings two template \api s for computing \dft s in one
    dimension, ie. one-time transforms and plans;
    \item the user can choose between \gsl, \fftw\ and our Boost licensed \bsl,
    backends;
    \item besides complex \dft s, the library provides a specialized interface
    for real numbers that sets the ground for the real transform optimization
    (usually a factor of 2 in computation and memory);
    \item in addition the \bsl\ backend is able to compute \dft\ in the more
    abstract sense of ring algebras leading to a higher class of \dft s like the
    \emph{Number Theoretical Transform};
    \item memory allocation can be completely controlled by the user (at least
    in the case of the \bsl\ backend) through the template parameter supplied
    allocator;
    \item the library also supplies a convolution function for complex
    sequences;
    \item the \bsl\ backend has theoretically optimal complexity for real and
    complex transforms $\Order(N \log N)$ for any size $N$, it is able to
    exploit the real transform optimization for composite sizes and its
    algebraic \dft\ engine has
    $\Order(N \log N)$ complexity for highly composite sizes (eg. powers of 2).
\end{itemize}
Some of the issues, features and improvements that the authors would like address
for future development are listed below:
\begin{itemize}
    \item add the documentation inside \boostmath\ webpage;
    \item optimize the \bsl\ backend to use less scratch memory and produce
    faster computation times;
    \item add support for multithreading;
    \item add support for multidimentional \dft\ transforms (possibly in
    coordination with the \boostublas\ library);
    \item work in collaboration with \boostmpi\ team to create distributed memory
    transforms and tacke the $D$-dimentional scalability problem proposing a
    $D-1$ domain decomposition similar to \cite{pippig_13};
\end{itemize}
