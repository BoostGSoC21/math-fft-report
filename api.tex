\section{Detailed API}
Boost.Math.FFT brings two kinds of interfaces: one-time and planned transforms.
The first one is inspired by the standard library's \texttt{std::transform} 
and the second by \fftw's plan interface. Listings
\ref{ls:complex_dft_transform} and \ref{ls:complex_dft_plan} are two examples of
the use of the one-time and planned interfaces respectively.

The planned transform is used with
the declaration of a plan variable and subsequently calling its methods.
The internal state of the plan is determined by the size of the \dft\ transform,
and throughout its life it caches the necessary data and memory allocation
needed to perform \dft s on request. In principle it should be faster to call a
method of a plan, than the equivalent routine from the one-time interface.

The library's interface can be further classified according to the domain where
the \dft\ is applied: let it be complex numbers, real and halfcomplex, or a
purely algebraic set satisfying the ring axioms. This distinction is necessary
because all three behave differently and the user expects different interfaces.

Consider the complex case. There are only two types of \dft\ we can perform on
a $\Complex^n$ vector, they are the \verb|forward| and the \verb|backward|, 
because there are only two roots of unity in that domain:
$w = \exp( - 2 \pi i/n)$ 
and
$w^{-1} = \exp( 2 \pi i/n)$.
While in general for an algebraic transform on a type that satisfies the ring
axioms the \dft\ engine has no way of knowing \emph{a priori} what is the root
of unity $w$ and that one must be supplied by the user either to the constructor
of the plan or as an extra argument to the one-time transform routine. In this
case, once $w$ is know there are again only two types of \dft s: \verb|forward| 
when we use $w$ or \verb|backward| when using $w^{-1}$.

The real/halfcomplex interface works very differently from the complex and the
algebraic. Since the result of any \dft\ on a $\Real^n$ vector yields always a
vector in $\hComplex^n$ and viceversa; this interface contains the methods:
\verb|real_to_halfcomplex| and \verb|halfcomplex_to_real| to emphasize that
property. Because of the trivial relations between the results of the forward
and backward transforms applied to real and/or halfcomplex data (see fig.
\ref{fig:halfcomplex}) we do not provide a choice for forward or backward
transform. \verb|real_to_halfcomplex| will in effect compute a forward \dft\ on
a $\Real^n$ vector and \verb|halfcomplex_to_real| will compute a backward \dft\ on a
$\hComplex^n$ vector. We also provide the methods \verb|real_to_complex|
and \verb|complex_to_real| similar to the previous \verb|real_to_halfcomplex|
and \verb|halfcomplex_to_real| respectively, where the halfcomplex part is
represented in an array of $n$ complex numbers, hence keeping redundant values.


%proposes the design and implementation of a modern C++ template
%interface for \dft\ within \boostmath\ library. The implementation will consist
%of a combination of routines written by the author based on well known \fft\
%algorithms and oriented to solve the \dft\ problem as presented in the
%Definition \ref{def:dft} for any data type that satisfies the Ring axioms, 
%and a back-end to the \fftw\ library for the special cases of
%the complex and real number types supported by \fftw. We aim to design an \api\
%that is flexible and powerful, so that the implementation can be adapted
%seamlessly to different back-ends without significant performance losses.

% \lstset{columns=flullflexible,basicstyle=\ttfamily}
% 
% A possible user interface for 1-dimensional \dft{}s could 
% consist of a set of overloads of a single function \texttt{dft}.
% Two of which could handle the general case of \dft\ as in Definition
% \ref{def:dft}: see
% Listings \ref{ls:dft1} for the in-place transformation 
% and Listings \ref{ls:dft2} for a transformation that is saved into a different memory
% location. The size of the \dft\ is deduced 
% as $n=$\texttt{std::distance(beg1,end1)} and the root of unity \texttt{w} for this
% size must be provided by the user. These routines can compute \dft{}s for any
% type that satisfy the Ring axioms as well as complex numbers.
% 
% \begin{lstlisting}[language=C++,label=ls:dft1,caption=General case of the DFT in-place.]
% template<class InputIterator, class T> 
% void 
% dft(InputIterator beg1, InputIterator end1, 
%     T w /* root unity */ );
% \end{lstlisting}
% 
% % out-of-place DFT, the size is deduced by std::distance(beg1,end1)
% % the root of unity is supplied by the user
% \begin{lstlisting}[language=C++,label=ls:dft2,caption=General case of the DFT out-of-place]
% template<class InputIterator, class OutputIterator, class T> 
% void 
% dft(InputIterator beg1, InputIterator end1,
%     OutputIterator beg2, 
%     T w /* root unity */ );
% \end{lstlisting}
% 
% \noindent Two other overloads of the complex \dft{}s are needed to release the user
% from the burden of computing the root of unity by himself and explicitely call
% the specialized back-end: see
% Listings \ref{ls:complex_dft1} and \ref{ls:complex_dft2}. The size of the \dft\
% is again deduced as $n=$\texttt{std::distance(beg1,end1)} and
% \texttt{complex\_dft}
% is an \texttt{enum class} that can have two states: \texttt{forward} or
% \texttt{backwards} that correspond to the roots of unity 
% $\exp(-i\, 2\pi/n)$
% and 
% $\exp(+i\, 2\pi/n)$ respectively,
% where $i=\sqrt{-1}$.
% % in-place complex DFT, the size is deduced by n=std::distance(beg1,end1)
% % the root of unity is deduced as exp(-I 2 pi/n) for t=forward and exp(I 2 pi/n)
% % for t = backwards
% \begin{lstlisting}[language=C++,label=ls:complex_dft1,caption=DFT for complex numbers in-place.]
% template<class InputIterator> 
% void
% dft(InputIterator beg1, InputIterator end1, 
%     complex_dft t);
% \end{lstlisting}
% 
% % out-of-place complex DFT, the size is deduced by n=std::distance(beg1,end1)
% % the root of unity is deduced as exp(-I 2 pi/n) for t=forward and exp(I 2 pi/n)
% % for t = backwards
% \begin{lstlisting}[language=C++,label=ls:complex_dft2,caption=DFT for complex numbers out-of-place.]
% template<class InputIterator, class OutputIterator> 
% void
% dft(InputIterator beg1, InputIterator end1, 
%     OutputIterator beg2, 
%     complex_dft t);
% \end{lstlisting}
% 
% \noindent Finally there will be another routine to compute convolutions: see
% Listings \ref{ls:convolution}. That computes the \cdc\ of size
% $n=$\texttt{std::distance(beg1,end1)}, using the root of unity \texttt{w} provided by the
% user or, if omited, deduced as $\exp(-i\, 2\pi/n)$ for complex and real types only.
% % Convolution, the size is deduced by n=std::distance(beg1,end1)
% % the root of unity is optional and can be internally deduced for T = Real or
% % complex<Real>
% \begin{lstlisting}[language=C++,label=ls:convolution,caption=Convolution routine.]
% template<class InputIterator, class OutputIterator, class T> 
% void
% convolution(InputIterator beg1, InputIterator end1, 
%             InputIterator beg2,
%             OutputIterator beg3, 
%             T w /* root of unity, optional */);
% \end{lstlisting}

%\subsection{Deliverables}
% This project proposes to deliver at the end of the \gsoc-2021 a C++ library for
% computing \fft{}s, with the following components:
% \begin{itemize}
%     \item A template user interface for 1-dimensional \dft.
%     \item A template user interface for computing convolutions.
%     \item A back-end that wraps \fftw\ routines to handle the standard complex types
%     ie. \verb|std::complex| instantiated on \verb|float|,
%     \verb|double| and \verb|long double|.
%     \item A general purpose back-end, based on 
%     templates routines for computing \dft\ using the algorithms
%     mentioned in the previous section based on the weaker notion of rings and
%     roots of unity. 
%     This back-end already provides support for \ntt\ without the need to write
%     specialized routines.
%     \item Specialization for complex to complex \dft, with the use of the \fftw\
%     back-end whenever possible,
%     and tested support for Boost Multiprecision types such as
%     \verb|cpp_complex|.
%     \item Specialization for real to complex \dft, with the use of the \fftw\
%     back-end whenever possible/
%     \item Unit tests and benchmarks.
%     \item Documentation with examples.
%     %\item support for $D$-dimensional \dft{}s,
%     %\item (if time permits) with multithreaded capabilities.
% \end{itemize}

\subsection{Future outlook}
Describe the pending issues and plans for improvement:
\begin{itemize}
    \item Documentation inside boost,
    \item optimization of the Boost routines,
    \item multithreading,
    \item multidimentional transforms,
    \item to work in collaboration with boost.mpi to create distributed memory
    transforms to tacke the $D$-dimentional scalability problem proposing a
    $D-1$ domain decomposition similar to \cite{pippig_13}.
\end{itemize}
