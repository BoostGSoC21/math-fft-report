\section{Quickstart}

Our interface was designed for easy of use. The user can start performing
complex FFTs by just calling one of the \texttt{std::transform}-like routines
\texttt{forward} or \texttt{backward} that correspond to a Discrete Fourier
Transform (DFT) in $\Complex^n$ space with the root of unity
$\exp(-\frac{2\pi}{n} i)$ and $\exp(\frac{2\pi}{n} i)$ respectively.
See listings \ref{ls:complex_dft_transform}
\begin{lstlisting}[language=C++,label=ls:complex_dft_transform,caption=Simple complex FFT
transform.]
using fft_transform = ::boost::math::fft::bsl_transform;
std::vector< std::complex<double> > A{1.,2.,3.,4.},B;
fft_transform::forward(A.cbegin(),A.cend(),std::back_inserter(B));
fft_transform::backward(B.cbegin(),B.cend(),B.begin());
\end{lstlisting}
The size of the transform $n$, is deduced from the input range and the type
complex representation is deduced from the input iterators.

If multiple transforms of the same size are required, it is recommended to cache
the engine internal state by using the plan-like interface.
\begin{lstlisting}[language=C++,label=ls:complex_dft_plan,caption=Simple complex FFT
Plan.]
using fft_plan = ::boost::math::fft::bsl_dft;
std::vector< std::complex<double> > A{1.,2.,3.,4.},B;

fft_plan<std::complex<double>> P(A.size());
P.forward(A.cbegin(),A.cend(),std::back_inserter(B));
P.backward(B.cbegin(),B.cend(),B.begin());
\end{lstlisting}
