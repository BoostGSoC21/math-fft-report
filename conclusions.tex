\section{Conclusions}
This \gsoc\ project has brought a modern C++ interface for Discrete Fourier
Transforms into \boostmath\ library.
The current design allows the user to select from three possible backends:
\gsl\, \fftw\ and the Boost licensed \bsl, all three with the same user
interface. The classes and functions are templates on the variable type, though
there are limits to the type of variables with the \gsl\ and \fftw\ backends.
For instance, \gsl\ supports only double precision standard complex,
while \fftw\ has support for single, double and quad precision standard complex.
The \bsl\ backend, on the other hand, has been written
completely in terms of templates, hence it works for any user provided type that
represents complex numbers. 
The novelty of our \dft\ design lies precisely in
the exploitation of the C++ templates to abstract away the details of the
underlying type. 
The reach of this approach goes beyond the complex numbers, in fact we were able
to write \fft s that work with any type that satisfy the algebraic ring axioms.
A possible application of non-complex \fft s that our design is able to produce
is the so called \emph{Number Theoretical Transform}, with possible application to the
fast multiplication of big integers.


There is still work to be done. The \api\ needs to be extended to include the very
often requested \dft\ in multiple dimensions and support for multi-processor
architectures. And the current implementation of the \bsl\ backend needs
optimization.
